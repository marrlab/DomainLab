\section{Summary}\label{sec:summary}
Deep learning (DL) models have been used in tackling real-world
challenges in various areas, such as computer vision and medical imaging
or computational pathology. However, while generalizing to unseen data
domains comes naturally to humans, it is still a significant obstacle
for machines. By design, most DL models assume that training and testing
distributions are well aligned, causing them to fail when this is
violated. Instead, domain generalization aims at training domain
invariant models that are robust to distribution shifts~\cite{wang2022generalizing}.

We introduce DomainLab, a Python package for domain generalization.
Compared to existing and concurrent solutions, DomainLab excels at the
extent of modularization by decoupling the various factors that
contribute to the performance of a domain generalization method: How the
domain invariant regularization loss is computed remain decoupled and
transparent to other factors like what neural network architectures are
used for each component, what transformations are used for observations
and how the neural network weights are updated. The user can mix and
match different combinations of the individual factors and evaluate the
impact on generalization performance.

Thanks to the modularized design of DomainLab, methods with complex
structure and components like some of the causal domain generalization
methods and generative model based methods~\cite{ilse2020diva, mahajan2021domain, sun2021hierarchical}, self-supervised learning based domain generalization method like~\cite{carlucci2019domain}, 
which do not exist in other existing and concurrent solutions, have been implemented, 
and can be easily integrated with adversarial methods~\cite{levi2021domain, ganin2016domain, akuzawa2020adversarial} and other training paradigms~\cite{rame2022fishr}.
We found it difficult to implement those aforementioned missing methods
into the codebase of existing and concurrent solutions due to
limitations in their code architecture designs.

DomainLab offers user functionality to specify custom datasets with
minimal configuration file without changing the codebase of DomainLab.

DomainLab follows software design pattern and is well documented and
tested.

DomainLab currently supports the PyTorch backend.

DomainLab's documentation is hosted at
\url{https://marrlab.github.io/DomainLab}, and its source code can be
found at \url{https://github.com/marrlab/DomainLab}.
